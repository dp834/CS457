\documentclass{article}

\usepackage{fullpage}
\usepackage{amsmath,amsfonts,amsthm,amssymb}

\begin{document}
\title{CS 457, Data Structures and Algorithms I\\
Third Problem Set}
\date{October 17, 2019}
\maketitle
\begin{center}
\textbf{Due on October 25. Collaboration is not allowed. Contact Daniel and me for questions.}
\end{center}
\begin{enumerate}


\item (11 pts) Prove tight \textbf{worst-case} asymptotic upper bounds for the following recurrence equation 
that depends on a variable $q\in [0, n/4]$. Note that you need to prove an upper bound that is true for every 
value of $q\in [0, n/4]$ and a matching lower bound for a specific value of $q\in [0, n/4]$ of your choosing. 
Do not assume that a specific $q$ yields the worst case input; instead, formally identify the $q$ which maximizes 
the running time. (Hint: look at the bottom of Page 180 for the analysis of the worst-case running time of Quicksort)
\begin{equation*}
T(n) =
\begin{cases}
1 &\text{if $n\leq 2$}\\
T(n-2q-1)+T(3q/2)+T(q/2)+\Theta(1) &\text{otherwise.}
\end{cases}
\end{equation*}


\item (24 pts) Given an array $S$ of $n$ distinct numbers provide $O(n)$-time algorithms for the following:
\begin{itemize}
\item Given two integers $k,\ell \in \{1,\dots, n\}$ such that $k\leq \ell$, find all the $i$th order statistics of $S$ for \emph{every} 
$i\in \{k, \dots, \ell\}$.

\item Given some integer $k\in \{1,\dots, n\}$, find the $k$ numbers in $S$ whose \emph{values} are closest to that of the median of $S$.
\end{itemize}



\item (25 pts) Consider the following silly randomized variant of binary search. You are given 
a sorted array $A$ of $n$ integers and the integer $v$ that you are searching for is chosen uniformly 
at random from $A$. Then, instead of comparing $v$ to the value in the middle
of the array, the randomized binary search variant chooses a random number $r$ from $1$ to $n$ and
it compares $v$ with $A[r]$. Depending on whether $v$ is larger or smaller, this process is
repeated recursively on the left sub-array or the right sub-array, until the location of $v$ is
found. Prove a tight bound on the expected running time of this algorithm.

\item (20 pts) You are given a set $S$ of $n$ integers, as well as one more integer $v$. 
\begin{itemize}
\item[a)] Design an algorithm that
determines whether or not there exist two distinct elements $x,y\in S$ such that $x+y=v$. Your algorithm
should run in time $O(n\log n)$, and it should return $(x,y)$ if such elements exist and $(NIL, NIL)$ otherwise.
\item [b)] Formally explain why your algorithm runs in $O(n\log n)$ time.
\end{itemize}

\item (20 pts) Suppose that you are given a sorted array $A$ of {\em distinct} integers $\{a_1, a_2,\dots, a_n\}$,
drawn from $1$ to $m$, where $m>n$. 
\begin{itemize}
\item[a)] Give an $O(\log n)$ algorithm to find an integer from $[1, m]$ that is not 
present in $A$. For full credit, find the smallest such integer. 
\item [b)] Formally explain why your algorithm runs in $O(\log n)$ time.
\end{itemize}

\end{enumerate}
\end{document}