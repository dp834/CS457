\documentclass{article}

\usepackage{fullpage}
\usepackage{amsmath,amsfonts,amsthm,amssymb}
\usepackage[noline,noend,linesnumbered]{algorithm2e}

\begin{document}
\title{CS 457, Data Structures and Algorithms I\\
Second Problem Set}
\date{October 8, 2019}
\maketitle
\begin{center}
\textbf{Due on October 16. Collaboration is \textbf{not} allowed. Contact Daniel and me for questions.}
\end{center}
\begin{enumerate}


\item (24 pts)
Solve the following recurrence equation (tight upper \emph{and} lower bounds!)
\begin{equation*}
T(n)=
\begin{cases}
1 &\text{if $n< 1$}\\
2T(n/2)+23 &\text{otherwise.}
\end{cases}
\end{equation*}

\begin{itemize}
\item [a)] Using the \emph{master theorem}. Make sure to explain which case applies and why.
\item [b)] Using the \emph{recursion tree} method. Do not use asymptotic notation for the depth of the recursion tree; use exact upper and lower bounds instead. 
\item [c)] Using the \emph{substitution} method. Make sure to show that the boundary conditions hold as well; choose the constants $c$ and $n_0$ appropriately.
\end{itemize}



\item (30 pts)
For the following algorithm calls, prove a {\em tight} asymptotic bound for their worst-case running time. Pay attention to the input in the algorithm \emph{calls}!

\begin{itemize}
\item [a)] The call to \textsc{Recursive-Algorithm}($n$) for some $n>1$.
\vspace{10pt}

\begin{algorithm}[H]
\textsc{Recursive-Algorithm}($a$)\;
$q=0$\;
\If{$a\geq 1$}{
	\For{$i=1$ \KwTo $\lfloor a\rfloor$}{q=q+1}
	\textsc{Recursive-Algorithm}($a/2$)\;
	\textsc{Recursive-Algorithm}($a/5$)\;
	\textsc{Recursive-Algorithm}($a/9$)\;

}
\end{algorithm}


\vspace{.3in}

\item[b)] The call to \textsc{Recursive-Algorithm2}($n,n$) for some $n>2$.
\vspace{10pt}

\begin{algorithm}[H]
\textsc{Recursive-Algorithm2}($a, b$)\;
\If{$a\geq 2$ ~and~ $b\geq 2$}{
	$u=a/3$\;
	$v=b-1$\;
	\textsc{Recursive-Algorithm2}($u, v$)\;
}
\end{algorithm}
\end{itemize}

\newpage
\item (30 pts) In solving the following recurrence equations, first try to use the master theorem. If it does not apply, explain why this is the case.

\begin{itemize}
\item [a)] Solve the following recurrence equation.
\begin{equation*}
T(n) = 
\begin{cases}
1 &\text{if $n\leq 1$}\\
T(2n/3)+T(n/4)+\Theta(n) &\text{otherwise.}
\end{cases}
\end{equation*}
\item [b)] Solve the following recurrence equation.
one
\begin{equation*}
T(n) = 
\begin{cases}
1 &\text{if $n\leq 1$}\\
4T(n/2)+n^2/\log n &\text{otherwise.}
\end{cases}
\end{equation*}

\item [c)] For the following equation provide an exact (\emph{not asymptotic}) closed form solution.
\begin{equation*}
T(n) = 
\begin{cases}
1 &\text{if $n\leq 1$}\\
5T(n/2) &\text{otherwise.}
\end{cases}
\end{equation*}
\end{itemize}


\item (16 pts) Consider the following variation of Merge Sort: rather than dividing the array into
two equal sized parts, recursively sorting each of them, and then merging them together, we divide
the array into $\sqrt{n}$ equal sized parts instead. Once we recursively sort each one of these
$\sqrt{n}$ parts of size $\sqrt{n}$ each, we need $\Theta(n\lg n)$ time in order to merge them together,
leading to the following recurrence equation:
\begin{equation*}
T(n)=
\begin{cases}
1 &\text{if $n\leq 1$}\\
\sqrt{n}T(\sqrt{n})+n\lg n &\text{otherwise.}
\end{cases}
\end{equation*}
Solve this recurrence equation, providing tight upper and lower bounds
\begin{itemize}
\item [a)] Using the \emph{recursion tree} method. 
\item [b)] Using the \emph{master theorem} after applying an appropriate change of variables.
\end{itemize}

\end{enumerate}
\end{document}