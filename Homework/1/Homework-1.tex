\documentclass{article}

\usepackage[margin=1in]{geometry}
\usepackage{listings}

\lstset{
    basicstyle=\ttfamily,
    mathescape
}

\author{Damien Prieur}
\title{Homework 1 \\ CS 457}
\date{}

\begin{document}

\maketitle

\section*{Question 1}
Find two functions $f(n)$ and $g(n)$ that satisfy the following relationship. If no such $f$ and $g$ exist, then try to explain why this is the case.
\begin{enumerate}
\item[a.] $f(n) \in o(g(n))$ and $f(n) \notin \Theta (g(n))$ \\ Let $f(n)=n$ and $g(n)=n^2$

\item[b.] $f(n) \in \Theta (g(n))$ and $f(n) \in o(g(n))$ \\ No such functions exist as $g(n)$ cannot be strictly larger than $f(n)$ asymptotically and also bound $f(n)$ from below.

\item[c.] $f(n) \in \Theta (g(n))$ and $f(n) \notin O(g(n))$ \\ No such functions exist as $f(n) \in \Theta (g(n))$ is only true if $f(n) \in O(g(n))$

\item[d.] $f(n) \in \Omega (g(n))$ and $f(n) \notin O(g(n))$ \\ Let $f(n)=n (1+\cos n)$ and $g(n)=1$

\end{enumerate}

\section*{Question 2}
For each of the following questions, briefly explain your answer.
\begin{enumerate}
\item[a.] If I prove that an algorithm takes $O(n^2)$ worst-case time, is it possible that it takes $O(n)$ time on some inputs? \\
Yes, your worst-case inputs does not give you an upper bound on your best case input.
See insertion sort from the lecutures, it's $O(n)$ on sorted inputs and $O(n^2)$ in it's worst case.

\item[b.] If I prove that an algorithm takes $O(n^2)$ worst-case time, is it possible that it takes $O(n)$ time on all inputs? \\
No because there is at least one input, your worst-case input, that will run in $O(n^2)$ time not $O(n)$.

\item[c.] If I prove that an algorithm takes $\Theta (n^2)$ worst-case time, is it possible that it takes $O(n)$ time on some inputs? \\
Yes, if your best case input is $O(n)$, see insertion sort again.

\item[d.] If I prove that an algorithm takes $\Theta (n^2)$ worst-case time, is it possible that it takes $O(n)$ time on all inputs? \\
No, because there is at least one input, your worst case input, that will run in at least $\Theta (n^2)$ time.

\end{enumerate}

\clearpage

\section*{Question 3}
Let $f(n)$ and $g(n)$ be any two functions with $f(n) > 1$ and $g(n) > 1$ for all $n$. Is it true that $15f(n)^2+23g(n)^2 \in \Theta((f(n)+g(n))^2)$? Give a formal justification for your answer.
We want to show

$$0 \leq c_{1}((f(n)+g(n))^2) \leq 15f(n)^2+23g(n)^2 \leq c_{2}((f(n)+g(n))^2)$$

for some

$$c_{1} > 0, c_{2} > 0, \forall n > n_{0}$$

We can start by expanding which gives us

$$0 \leq c_{1}(f(n)^2+2g(n) \cdot f(n) + g(n)^2) \leq 15f(n)^2+23g(n)^2 \leq c_{2}(f(n)^2+2g(n) \cdot f(n) + g(n)^2)$$

I am going to ignore the $0 \leq $ case as it is given that both functions are greater.
We must solve each inequality on it's own, we will start with the first. and divide by $c_{1}$

$$f(n)^2+2g(n) \cdot f(n) + g(n)^2 \leq \frac{15}{c_{1}}f(n)^2+\frac{23}{c_{1}}g(n)^2$$

Now we can subtract the $f(n)^2$ and $g(n)^2)$ terms which gives

$$2g(n) \cdot f(n) \leq (\frac{15}{c_{1}}-1)f(n)^2+(\frac{23}{c_{1}}-1)g(n)^2$$

Now we can divide by $2g(n) \cdot f(n)$ to get

$$1 \leq (\frac{15}{c_{1}}-1)\frac{f(n)}{2g(n)}+(\frac{23}{c_{1}}-1)\frac{g(n)}{2f(n)}$$

Now we need to know something about these function specifically what their ratio is.
We can analyze this by using cases and looking at large enough n where the cases will be one of three things

\begin{enumerate}

\item $$\lim_{n \to \infty} \frac{f(n)}{g(n)} = \infty$$
\item $$\lim_{n \to \infty} \frac{f(n)}{g(n)} = 0$$
\item $$\lim_{n \to \infty} \frac{f(n)}{g(n)} = \frac{a}{b}$$ Where $a$ and $b$ are integers

\end{enumerate}

Looking at each case we get

\begin{enumerate}

\item $$1 \leq (\frac{15}{c_{1}}-1) \infty + (\frac{23}{c_{1}}-1)\cdot 0$$
$$1 \leq (\frac{15}{c_{1}}-1) \infty$$ \\
Which holds as long as $(\frac{15}{c_{1}}-1) > 0$ So solving for $c_{1}$
$$c_{1} < 15$$

\item $$1 \leq (\frac{15}{c_{1}}-1)\cdot 0  + (\frac{23}{c_{1}}-1)\infty$$
$$1 \leq (\frac{23}{c_{1}}-1)\infty$$
Which also holds as long as $(\frac{23}{c_{1}}-1) > 0$ So solving for $c_{1}$
$$c_{1} <23$$

\item $$1 \leq (\frac{15}{c_{1}}-1)\frac{a}{b}  + (\frac{23}{c_{1}}-1)\frac{b}{a}$$
Solving for $c_{1}$ we get
$$c_{1} \leq \frac{15a^2+23b^2}{a^2+ab+b^2}\qquad \forall a>0, \forall b>0$$

\end{enumerate}

Now we must look at the second inequality and follow the same steps

$$\frac{15}{c_{2}}f(n)^2+\frac{23}{c_{2}}g(n)^2 \leq f(n)^2+2g(n) \cdot f(n) + g(n)^2$$

Now we can subtract the $f(n)^2$ and $g(n)^2)$ terms which gives

$$(\frac{15}{c_{2}}-1)f(n)^2+(\frac{23}{c_{2}}-1)g(n)^2 \leq 2g(n) \cdot f(n)$$

Now we can divide by $2g(n) \cdot f(n)$ to get

$$(\frac{15}{c_{2}}-1)\frac{f(n)}{2g(n)}+(\frac{23}{c_{2}}-1)\frac{g(n)}{2f(n)} \leq 1$$

Using the same three cases as before we get

\begin{enumerate}

\item $$(\frac{15}{c_{2}}-1) \infty + (\frac{23}{c_{2}}-1)\cdot 0 \leq 1$$
$$(\frac{15}{c_{2}}-1) \infty \leq 1$$ \\
Which holds as long as $(\frac{15}{c_{2}}-1) \leq 0$ So solving for $c_{2}$
$$c_{2} \geq 15$$

\item $$(\frac{15}{c_{2}}-1)\cdot 0  + (\frac{23}{c_{2}}-1)\infty \leq 1$$
$$(\frac{23}{c_{2}}-1)\infty \leq 1$$
Which also holds as long as $(\frac{23}{c_{2}}-1) \leq 0$ So solving for $c_{2}$
$$c_{2} \geq 23$$

\item $$(\frac{15}{c_{2}}-1)\frac{a}{b}  + (\frac{23}{c_{2}}-1)\frac{b}{a} \leq 1$$
Solving for $c_{2}$ we get
$$c_{2} \geq \frac{15a^2+23b^2}{a^2+ab+b^2}\qquad \forall a>0, \forall b>0$$

\end{enumerate}

Since we can always find a $c_{1}$ and $c_{2}$ our original inequality holds. So
$$15f(n)^2+23g(n)^2 \in \Theta((f(n)+g(n))^2)$$
$$QED$$

\section*{Question 4}
Is the following statement {\bf true} or {\bf false}, and why? \\
If $f(n) \in O(g(n))$ then $2^{f(n)} \in O(2^{g(n)})$. \\
In order to be big-Oh we must show
$$2^{f(n)} \leq c2^{g(n)}\qquad \forall n>n_{0}$$
$$\log_{2}(2^{f(n)}) \leq \log_{2}(c2^{g(n)})$$
$$f(n) \leq \log_{2}c + g(n)$$
$$f(n) - g(n) \leq \log_{2}c $$
Since $f(n) \in O(g(n))$ we know eventually $g(n) > f(n)\quad \forall n>n_{0}$
This tells us that
$$f(n)-g(n) < 0\qquad \forall n>n_{0}$$
Which gives us
$$f(n)-g(n)<0<\log_{2}c\qquad \forall n>n_{0}$$
$$c>0$$
$$QED$$

\section*{Question 5}
What value is returned by the following function? Express your answer as a function of $n$.
From this, deduce the worst-case running time using Big Oh notation.
For full credit, provide a lower bound as well, thus leading to a bound with $\Theta(\cdot)$ notation.
Then, also deduce the worst-case running time if we change the while loop clause to $j \leq i$ instead of $j \leq n$.
\begin{lstlisting}
function fool(n)
r := 0
for i := 1 to n do
    j:= 1
    while j $\leq$ n do
        r := r + 1
        j := 2$\cdot$j
return r
\end{lstlisting}

This functions returns $n*(1+\lfloor \log_{2}n \rfloor)$ which can also be written as $n*\lceil \log_{2}n \rceil$ so we have:
$$fool(n)=n\lceil \log_{2}n \rceil$$
To find the runtime of this we can look at each line and determine it's runtime.

\begin{enumerate}

\item $c_{1}$
\item $c_{2}n$
\item $c_{3}n$
\item $c_{4}n\lceil \log_{2}n \rceil$
\item $c_{5}n\lceil \log_{2}n \rceil$
\item $c_{6}n\lceil \log_{2}n \rceil$
\item $c_{7}$

\end{enumerate}
Adding all these terms up we get
$$T(n) = c_{1}+c_{7} + n(c_{2} + c_{3}+\lceil \log_{2}n \rceil(c_{4} + c_{5} + c_{6}))$$
$T(n)$ can be bounded above by removing the ceiling with a plus one. This would make it strictly larger.
$$T(n) = c_{1}+c_{7} + n(c_{2} + c_{3}+\log_{2}(2n)(c_{4} + c_{5} + c_{6}))$$
Combine constants to get
$$T(n) = C_{1} + C_{2}n+C_{3}n\log_{2}(2n)$$
Which we can see is $O(n \log n)$ as well as $\Omega(n \log n)$ in the worst case
$$T(n) \in \Theta (n \log n)$$

If the loop's condition was changed to $ j \leq i$ lines 4-6 would need to change and would run according to this formula
$$\sum_{i=1}^{n}\lceil \log_{2}i \rceil$$
Worst case again would be taking the log and adding 1
$$\sum_{i=1}^{n} (1 + \log_{2}(i)) = n + \log_{2}(n!)$$

Which gives us these new runtimes for each line

\begin{enumerate}

\item $c_{1}$
\item $c_{2}n$
\item $c_{3}n$
\item $c_{4}(n + \log_{2}(n!))$
\item $c_{5}(n + \log_{2}(n!))$
\item $c_{6}(n + \log_{2}(n!))$
\item $c_{7}$

\end{enumerate}
The worst case running time would be

$$T(n) = c_{1}+c_{7} + n(c_{2} + c_{3}+\log_{2}(n!)(c_{4} + c_{5} + c_{6}))$$
$$T(n) = C_{1} +C_{2}n + C_{3}n\log_{2}(n!)$$

\section*{Question 6}
What value is returned by the following function? Express your answer as a function of $n$.
From this, deduce the worst-case running time using Big Oh notation. For full credit, provide a lower bound as well, thus leading to a bound with $\Theta( \cdot )$ notation.

\begin{lstlisting}
function foo2(n)
r := 0
j := n
while j $\geq$ 1 do
    for i := 1 to j do
        r := r + 1
    j := j/2
return r

\end{lstlisting}

This function returns $ \sum_{i=1}^{\lfloor \log_{2}n \rfloor} (\lfloor \frac{n}{2^{n-1}} \rfloor - 1) $ which can be simplified to
$$ \sum_{i=1}^{\lfloor \log_{2}n \rfloor} (\lfloor \frac{n}{2^{n-1}} \rfloor) - \lfloor \log_{2}n \rfloor $$

To determine the runtime we can look at each line again

\begin{enumerate}

\item $c_{1}$
\item $c_{2}$
\item $c_{3}\lfloor \log_{2}n \rfloor$
\item $c_{4}\lfloor \log_{2}n \rfloor \sum_{i=1}^{j}1$
\item $c_{5}\lfloor \log_{2}n \rfloor \sum_{i=1}^{j}1$
\item $c_{6}\lfloor \log_{2}n \rfloor$
\item $c_{7}$

\end{enumerate}
Since we only care about worst case runtime we can remove the floors and add 1 to get
$$T(n) = c_{1} + c_{2} + c_{7} + \log_{2}n(c_{3}+c_{6} + (c_{4} +c_{5})(\sum_{j=1}^{\log_{2}(2n)}\sum_{i=1}^{j}1))$$
Combining constants we get
$$T(n) = C_{1} + \log_{2}n(C_{2} + C_{3}(\sum_{j=1}^{\log_{2}(2n)}\sum_{i=1}^{2^j}1))$$
Simplifying the sum
$$\sum_{j=1}^{\log_{2}(2n)}\sum_{i=1}^{2^j}1 = \sum_{j=1}^{\log_{2}(2n)}2^j = 4n - 2$$
Which means our worst-case run time is
$$T(n) = C_{1} + \log_{2}n(C_{2} + C_{3}(4n-2))$$
$$T(n) = C_{1} + C_{2}\log_{2}n + C_{3}(4n-2)\log_{2}n$$
Which we can see is $O(nlogn)$ as well as $\Omega(nlogn)$ so we have
$$ T(n) \in \Theta(nlogn)$$

\section*{Question 7}
Show that, if $c$ is a positive real number, then $g(n) = \sum_{i=0}^{n} c^i$
\begin{enumerate}

\item[a.] $\Theta(1)$ if $c < 1$
\item[b.] $\Theta(n)$ if $c = 1$
\item[c.] $\Theta(c^n)$ if $c > 1$

\end{enumerate}

First we have a closed form for our sum
$$ g(n) = \sum_{i=0}^{n} c^i = \frac{c^{n+1}-1}{c-1} \qquad c \ne 1$$
$$ g(n) = \sum_{i=0}^{n} c^i = n + 1 \qquad c = 1 $$
\begin{enumerate}

\item[a.] $$ k_{1} \leq \frac{c^{n+1}-1}{c-1} \leq k_{2} \qquad c < 1 $$
Since $c^{n+1} - 1 < c -1 \qquad c<1$ then $0 < \frac{c^{n+1}-1}{c-1} < 1$
So we know what $k_{1}$ and $k_{2}$ have to be
$$ k_{1}=0+\epsilon; k_{2}=1$$
Where $\epsilon$ is arbitrarily small

\item[b.] $$ k_{1}n \leq n + 1 \leq k_{2}n \qquad c = 1 $$
This holds true when
$$ k_{1}<=1 \qquad k_{2}>=2 $$



\item[c.] $$ k_{1}c^n \leq \frac{c^{n+1}-1}{c-1} \leq k_{2}c^n \qquad c > 1 $$
$$  k_{1}(c^{n+1}-1) \leq c^{n+1}-1 \leq k_{2}(c^{n+1}-1) \qquad c > 1 $$
Holds true when
$$ k_{1} <=1 \qquad k_{2} >= 2$$



\end{enumerate}

\end{document}
